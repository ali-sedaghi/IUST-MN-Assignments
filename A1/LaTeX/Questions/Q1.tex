\Problem
{روند تخصیص فرکانس}
{
طیف فرکانس رادیویی منبعی محدود است که متعلق به کل بشریت است. با توجه به پیشرفت سریع شبکه‌های تلفن همراه تقاضا برای آن نیز به طرز قابل توجهی یافته است.
این فرکانس توسط کنوانسیونی بین‌المللی به نام مقررات رادیویی تنظیم می‌شود. این کنوانسیون یک چارچوب اساسی درباره ویژگی‌های فرکانس رادیویی ساخته است که تمامی کشورها باید از این چارچوب پیروی کنند. توزیع فرکانس در مناطق مختلف به کشورها سپرده شده است. در صورت داشتن مجوز از دولت، مالکیت فرکانس رادیویی احراز می‌شود.
این کنوانسیون تلاش می‌کند مشکلاتی مانند تداخل، بهینه‌سازی استفاده از طیف، معرفی فناوری جدید، هماهنگی دولت‌ها و کشورهای همسایه را حل کند.
دولت‌ها بایستی از هزینه اختصاص فرکانس برآوردی داشته باشند.، تجهیزات را تایید کنند، مجوزهای لازم را صادر کنند.
یکی دیگر از وظایف مهم دولت‌ها هماهنگی با کشورهای همسایه است. زیرا این طیف فرکانس مرز جغرافیایی ندارد و باید از تداخل آن جلوگیری کرد.
دولت‌ها بایستی فرکانس‌های تخصیص یافته را به اطلاع
\lr{ITU}
برسانند.
همچنین ارتباطات خارجی با کنوانسیون‌های منطقه‌ای نیز بر عهده دولت‌ها می‌باشد و بایستی هماهنگی‌های لازم صورت گیرد.

\lr{ITU}
با برگذاری کنفرانس
\lr{WRC}
در هر سه یا چهار سال یکبار، درباره موارد ذکر شده صحبت می‌کند.

نحوه تقسیم و تخصیص فرکانس رادیویی به صورت سلسله مراتبی می‌باشد.
یعنی ابتدا
\lr{ITU}
این طیف را به قسمت‌های کوچک‌تر تقسیم می‌نماید و هر قسمت مختص کاربردی است.
سپس هر کشور تقسیم‌های مربوط به خود را انجام می‌دهد.
}
