\Problem
{نقش و جایگاه شرکت‌ها}
{
شرکت
\lr{Qualcomm}
که به صورت کوآلکام تلفظ می‌شود، یک شرکت بین‌المللی است که مقر اصلی آن در آمریکا می‌باشد.
این شرکت در توسعه نسل‌های مختلف نقش بسیاری داشته است.
برای مثال این شرکت در نسل 2 تلفن همراه تکنیک
\lr{CDMA}
را توسعه داد و جایگزینی برای
\lr{TDMA}
معرقی کرد.
این شرکت در نسل 3 نیز به دلیل دارا بودن تکنیک
\lr{CDMA}
بسیاری از شرکت‌ها را جذب خود کرد.
برای مثال شرکت
\lr{Nokia}
با این شرکت قراردادی بست.
چندین سال بعد نیز شرکت کوآلکام نسخه قبلی را توسعه داد و گسترده تر کرد و تکنیک
\lr{WCDMA}
را معرفی نمود.

\lr{Snapdragon}
گونه‌ای از
\lr{System On Chip}
به اختصار
\lr{SoC}
ها را تولید می‌کند.
از این سیستم‌ها در تلفن همراه و جاهایی که مصرف برق کمی می‌خواهیم بسیار استفاده می‌شود.
پردازنده این سیستم‌ها از خانواده
\lr{ARM}
می‌باشد.

در سال 2000 این دو شرکت به یکدیگر پیوسته شدند و به یک شرکت واحد کوآلکام تبدیل شدند.
این شرکت محصولات زیادی در حوزه تلفن همراه و مودم تولید می‌کند.

شرکت‌های دیگر نظیر سامسونگ، نوکیا و ...
با خرید چیپ‌های شرکت
\lr{Qualcomm Snapdragon}
و استفاده از آن‌ها در محصولات خود تلفن‌های همراه را تولید می‌کنند.
شرکت‌هایی نظیر سامسونگ و نوکیا خود در تولید این چیپ‌ها ناتوانند.
}
